\documentclass[12pt]{article}

\usepackage{sbc-template}
\usepackage{graphicx,url}
\usepackage[utf8]{inputenc}
\usepackage[brazil]{babel}
\usepackage[T1]{fontenc}
\usepackage[protrusion=true,expansion=true,final,babel]{microtype}

     
\sloppy

\title{Acerca da melhora de um \textit{pipeline} de produção de pipoca}

\author{Giovanne César O. de Souza\inst{1}, Guilherme Eiji E. Hantani\inst{1},\\Eduardo Gil S. Cardoso\inst{1}, Igor Matheus S. Moreira\inst{1}}


\address{
    Faculdade de Computação -- Instituto de Ciências Exatas e Naturais\\
    Universidade Federal do Pará -- Rua Augusto Corrêa, 01 -- 66.075-110\\
    Belém -- Pará -- Brasil
    \email{\{giovanne.souza,guilherme.hantani\}@icen.ufpa.br}
    \vspace{-10pt}
    \email{\{eduardo.serrao.cardoso,igor.moreira\}@icen.ufpa.br}
}

\begin{document} 

\maketitle

\begin{abstract}
    In this paper, which doubles as a deliverable for the Discrete Simulation discipline, a Kaggle data set that describes a popcorn production pipeline is exposed. A discrete simulation is created based on it, wherein different scenarios are created in hopes of streamlining and accelerating the process. [ELABORATE ON THE RESULTS, DO NOT EXCEED 10 LINES]
\end{abstract}
     
\begin{resumo}
    Neste artigo, que também é um entregável para a disciplina Simulação Discreta, um conjunto de dados que descreve um \textit{pipeline} de produção de pipoca é exposto. Uma simulação discreta é criada baseada nele, em que diferentes cenários são criados na esperança de simplificar e acelerar o processo. [ELABORAR SOBRE OS RESULTADOS, NÃO EXCEDER 10 LINHAS]
\end{resumo}



\bibliographystyle{sbc}
\bibliography{sbc-template}

\end{document}
